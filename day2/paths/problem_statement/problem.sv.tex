\ifx\boi\undefined\documentclass[swedish]{templates/boi}
\problemlanguage{.sv}
\newcommand{\practicesession}{Övningssession}
\newcommand{\dayone}{Dag 1}
\newcommand{\daytwo}{Dag 2}
\fi
\def\version{jury-1}
\problemname{Vägar}
En {\em graf} är en matematisk struktur som består av en uppsättning {\em noder}, och en uppsättning {\em bågar}, där varje båge sammanbinder två noder. Ett exempel på en graf med $4$ noder och $3$ bågar visas i exempelförklaringen nedan.

En {\em väg} i grafen definieras som en ordnad lista med $2$ eller fler noder, där det finns en båge mellan den första och andra noden i listan, mellan den andra och tredje, o.s.v. genom hela listan. Vi är bara intresserade av {\em enkla vägar} som inte innehåller samma nod flera gånger. Notera att listan är ordnad; exempelvis betraktas ``\texttt{5-6-7}'', ``\texttt{5-7-6}'' och ``\texttt{7-6-5}'' som olika vägar.

I den här uppgiften har varje nod i grafen en av $K$ färger. Uppgiften är att hitta antalet möjliga (enkla) vägar i vilka alla noder har olika färg.

\section*{\inputsection}
Första raden innehåller tre heltal: $N$ (antalet noder), $M$ (antalet bågar), och $K$ (antalet olika färger).

Andra raden innehåller $N$ heltal mellan $1$ och $K$ -- färgen på varje nod (första talet beskriver nod nummer $1$ och sista talet nod nummer $N$). 

Var och en av de följande $M$ raderna beskriver en båge och innehåller två heltal $a, b$ ($1 \le a, b \le N, a \neq b$) -- de två noderna som bågen sammanbinder. Det finns aldrig mer än en båge mellan samma två noder.

\section*{\outputsection}
Skriv ut ett heltal -- antalet vägar vars samtliga noder har olika färger. Antalet kommer alltid att vara lägre än $10^{18}$.

\section*{\constraints}
\testgroups

\noindent
\begin{tabular}{| l | l | l |}
\hline
\group & \points & \limitsname \\ \hline
1      & 23      & $1 \le N, M \le 100, 1 \le K \le 4$ \\ \hline
2      & 20      & $1 \le N, M \le 300\,000, 1 \le K \le 3$ \\ \hline
3      & 27      & $1 \le N, M \le 300\,000, 1 \le K \le 4$ \\ \hline
4      & 30      & $1 \le N, M \le 100\,000, 1 \le K \le 5$ \\ \hline
\end{tabular}

\section*{\sampleexplanation{1}}

\includegraphics[width=5cm]{pathsfig.pdf}

Grafen i första exemplet visas i figuren, där varje nod har färgats vit (färg 1), grå (färg 2) eller svart (färg 3). Det finns 10 vägar vars noder har olika färger: ``\texttt{1-2}'', ``\texttt{2-1}'', ``\texttt{2-3}'', ``\texttt{3-2}'', ``\texttt{2-4}'', ``\texttt{4-2}'', ``\texttt{1-2-4}'', ``\texttt{4-2-1}'', ``\texttt{3-2-4}'' och ``\texttt{4-2-3}''.

Notera att ``\texttt{1}'' inte är en godkänd väg eftersom det är en ensam nod, och inte heller ``\texttt{1-2-3}'', eftersom den innehåller två noder med färg $1$.

