\ifx\boi\undefined\ifx\problemname\undefined
\providecommand\sampleinputname{}
\providecommand\sampleoutputname{}
\documentclass[icelandic]{templates/boi}
\problemlanguage{.is}
\fi
\newcommand{\boi}{Eystrasaltsólympíuleikarnir í forritnu}
\newcommand{\practicesession}{Æfingakeppni}
\newcommand{\contestdates}{27. apríl - 1. maí, 2018}
\newcommand{\dayone}{Dagur 1}
\newcommand{\daytwo}{Dagur 2}
\newcommand{\licensingtext}{This problem is licensed under CC BY-SA 4.0.}
\newcommand{\problem}{Verkefni}
\newcommand{\inputsection}{Inntak}
\newcommand{\outputsection}{Úttak}
\newcommand{\interactivity}{Gagnvirkni}
\newcommand{\grading}{Flokkun}
\newcommand{\scoring}{Stigagjöf}
\newcommand{\constraints}{Takmarkanir}
\renewcommand{\sampleinputname}{Sýnidæmis inntak}
\renewcommand{\sampleoutputname}{Sýnidæmis úttak}
\newcommand{\sampleexplanation}[1]{Útskýring á sýnidæmi #1}
\newcommand{\sampleexplanations}{Útskýringar á sýnidæmum}
\newcommand{\timelimit}{Tímamörk}
\newcommand{\memorylimit}{Minnismörk}
\newcommand{\seconds}{s}
\newcommand{\megabytes}{MB}
\newcommand{\group}{Hópur}
\newcommand{\points}{Stig}
\newcommand{\limitsname}{Takmarkanir}
\newcommand{\additionalconstraints}{Auka takmarkanir}
\newcommand{\testgroups}{
Lausnin þín verður prófuð á einhvern fjölda prufuhópa, hver hópur gefur einhvern fjölda stiga.
Hver hópur inniheldur einhvern fjölda prufutilvika.
Til að fá stig fyrir hóp þarftu að leysa öll prufutilvik innan hópsins.
}
\fi
\def\version{jury-1}
\problemname{Leiðir}
{\em Net} er stærðfræðilegt mynstur sem er samansett af mengi hnúta og mengi leggja, þar sem hver leggur tengir tvo hnúta. Dæmi um net með $4$ hnútum og $3$ leggjum er sýnt í útskýringu sýnidæma að neðan.

{\em Leið} i neti er skilgreint sem raðaður listi af $2$ eða fleiri hnútum, þannig að það sé leggur milli samliggjandi hnúta í listanum.
Í þessu verkefni höfum við aðeins áhuga á {\em einföldum leiðum} þar sem enginn hnútur kemur fyrir oftar en einu sinni. Taktu eftir
að listinn er raðaður, til dæmis, ``\texttt{5-6-7}'', ``\texttt{5-7-6}'' og ``\texttt{7-6-5}'' eru mismunandi leiðir.

%For example, a road map could be a graph, if the vertices are towns or other places and the edges are roads that directly connect two places.

Í þessu verkefni er hver hnútur í netinu litaður með einum a $K$ litum. Verkefnið er að finna fjölda
mögulegra (einfaldra) leiða þar sem engir tveir hnútar eru litaðir eins.

\section*{\inputsection}
Fyrsta línan af inntakinu inniheldur þrjár heiltöllur: $N$ (fjöldi hnúta), $M$ (fjöldi leggja), og $K$ (fjöldi mismunandi lita).

%($1 \le N, M \le 3 \cdot 10^5, 1 \le K \le 5$).

Önnur línan af inntakinu inniheldur $N$ heiltölur á milli $1$ og $K$ -- litirnir á hverjum hnút (byrjar á hnút $1$ og endar á hnút $N$).

Næst koma $M$ línur, hver og ein þeirra lýsir legg og inniheldur tvær heiltölur $a, b$ ($1 \le a, b \le N, a \neq b$) -- hnútarnir tveir eru tengdir með leggnum.
Fyrir hverja tvo hnúta er mesta lagi einn leggur á milli þeirra.

\section*{\outputsection}
Skrifaðu út eina heiltölu -- fjölda leiða þar sem allir hnútarnir hafa mismunandi liti. Þessi tala verður alltaf minni en $10^{18}$.

\section*{\constraints}
\testgroups

\noindent
\begin{tabular}{| l | l | l |}
\hline
\group & \points & \limitsname \\ \hline
1      & 23      & $1 \le N, M \le 100, 1 \le K \le 4$ \\ \hline
2      & 20      & $1 \le N, M \le 300\,000, 1 \le K \le 3$ \\ \hline
3      & 27      & $1 \le N, M \le 300\,000, 1 \le K \le 4$ \\ \hline
4      & 30      & $1 \le N, M \le 100\,000, 1 \le K \le 5$ \\ \hline
\end{tabular}

\section*{\sampleexplanation{1}}

\includegraphics[width=5cm]{pathsfig.pdf}

Netið í fyrsta sýnidæminu er sýnt á myndinni, þar sem hver hnútur hefur verið litaður með hvítum (litur 1) , gráum (litur 2) eða svörtum (litur 3) lit.
Það eru 10 leiðir þar sem allir hnútarnir hafa mismunandi liti: ``\texttt{1-2}'', ``\texttt{2-1}'', ``\texttt{2-3}'', ``\texttt{3-2}'', ``\texttt{2-4}'', ``\texttt{4-2}'', ``\texttt{1-2-4}'', ``\texttt{4-2-1}'', ``\texttt{3-2-4}'' og ``\texttt{4-2-3}''.

Taktu eftir að ``\texttt{1}'' er ekki leyft sem leið, því það er aðeins einn hnútur. 
``\texttt{1-2-3}'' er einnig ekki leyft því það inniheldur tvær nóður með lit $1$.
