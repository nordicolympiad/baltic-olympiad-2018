\ifx\boi\undefined\ifx\problemname\undefined
\providecommand\sampleinputname{}
\providecommand\sampleoutputname{}
\documentclass[icelandic]{templates/boi}
\problemlanguage{.is}
\fi
\newcommand{\boi}{Eystrasaltsólympíuleikarnir í forritnu}
\newcommand{\practicesession}{Æfingakeppni}
\newcommand{\contestdates}{27. apríl - 1. maí, 2018}
\newcommand{\dayone}{Dagur 1}
\newcommand{\daytwo}{Dagur 2}
\newcommand{\licensingtext}{This problem is licensed under CC BY-SA 4.0.}
\newcommand{\problem}{Verkefni}
\newcommand{\inputsection}{Inntak}
\newcommand{\outputsection}{Úttak}
\newcommand{\interactivity}{Gagnvirkni}
\newcommand{\grading}{Flokkun}
\newcommand{\scoring}{Stigagjöf}
\newcommand{\constraints}{Takmarkanir}
\renewcommand{\sampleinputname}{Sýnidæmis inntak}
\renewcommand{\sampleoutputname}{Sýnidæmis úttak}
\newcommand{\sampleexplanation}[1]{Útskýring á sýnidæmi #1}
\newcommand{\sampleexplanations}{Útskýringar á sýnidæmum}
\newcommand{\timelimit}{Tímamörk}
\newcommand{\memorylimit}{Minnismörk}
\newcommand{\seconds}{s}
\newcommand{\megabytes}{MB}
\newcommand{\group}{Hópur}
\newcommand{\points}{Stig}
\newcommand{\limitsname}{Takmarkanir}
\newcommand{\additionalconstraints}{Auka takmarkanir}
\newcommand{\testgroups}{
Lausnin þín verður prófuð á einhvern fjölda prufuhópa, hver hópur gefur einhvern fjölda stiga.
Hver hópur inniheldur einhvern fjölda prufutilvika.
Til að fá stig fyrir hóp þarftu að leysa öll prufutilvik innan hópsins.
}
\fi
\def\version{jury-1}
\problemname{Skiptistraumur}
Fredrik er heima að leika sér með sérsmíðuðu lestarteina módeli sem hann er mjög stoltur af.
Lestarteinarnir samanstanda af $N$ hlutum sem eru tengdir í hring, númeraðir $1, 2, \dots, N$ réttsælis.
Rafstraumur í lestina fer í gegnum $M$ bogna víra sem liggja að hringnum. Hver hluti hefur að minnsta kosti einn vír við sig.

Hinsvegar hefur Fredrik fengið leið á lestinni sinni sem fer bara í hring og hefur ákveðað að bæta við 
\emph{lestrofa} á hvern hluta sem hann getur notað til að valda því að lestin fari af lestarteinunum og
í aðrar spennandi aðstæður. En rofarnir þurfa rafstraum. Ekki bara einhvern rafstraum, heldur þurfa þeir
sérstaklega \emph{skiptistraum}.\footnote{Þetta er skiljanlegt því lestarteinarnir eru sænskir -- í Svíþjóð
nota allir lestrofar (``växler'') skiptistraum (``växelström'').}
% Sorry about the terrible pun. If it translates into your language you can remove the footnote.

Fredrik hugsar að leiðin til að fá skiptistraum er að leiða rafstraum í báðar áttir.
Hver vír leiðir straum í eina átt (annað hvort réttsælis eða rangsælis) en Fredrik
ræður í hvora átt hver vír leiðir straum. Því vill hann ákveða fyrir hvern vír í hvaða
átt sá vír leiðir straum þannig að hver hluti lestarteinanna sé með réttsælis snúinn vír og rangsælis snúinn vír.

Geturðu hjálpað Fredrik?

\vspace{2mm}
%\hspace*{2mm}
\begin{center}
\includegraphics[width=0.5\textwidth]{alternatingfig.pdf}
\end{center}
\vspace{1mm}
{\em Lausn við fyrsta sýnidæminu. Bognu örvarnar fyrir utan teinana tákna vírana sem leiða rafstraum. Áttin sem örin bendir í táknar hvaða átt Fredrik valdi fyrir vírinn (þar sem litirnir skýra líka áttirnar). Taktu eftir að snúa má öllum örvunum við til að fá hina gildu lausnina: \texttt{11010}.}

\section*{\inputsection}
Fyrsta línan inniheldur tvær heiltölur $N$ og $M$, fjöldi hluta í lestarteinunum og fjöldi víra.

Næstu $M$ línur innihalda hver tvær heiltölur $1 \le a, b \le N$, sem þýðir að það vírinn
liggur við hluta $a, a+1, \dots, b$. Ef $b$ er minna en $a$ þá fer runan fram yfir, þ.e. hlutar $a, \dots, N, 1, \dots, b$ eru tengdir.
Taktu eftir að ef $a=b$ þá er aðeins einn hluti í rununni.

\section*{\outputsection}
Skrifaðu út línu með $M$ stöfum, hver annað hvort \texttt{0} eða \texttt{}. Stafur 
númer $i$ í línunni skal vera \texttt{0} ef straumurinn á vír númer $i$ á að streyma 
réttsælis, annars \texttt{1} ef hann á að streyma rangsælis. Ef það eru margar lausnir
til máttu skrifa út einhverja þeirra.

Ef það eru ekki til neinar gildar lausnir skaltu skrifa út ``\texttt{impossible}''.

\section*{\constraints}
\testgroups

\noindent
\begin{tabular}{| l | l | l | l |}
\hline
\textbf{\group} & \textbf{\points} & \textbf{\limitsname} & \textbf{\additionalconstraints} \\ \hline
  1     & 13     & $2 \le N, M \le 15$ & \\ \hline
  2     & 20     & $2 \le N, M \le 100$ & \\ \hline
  3     & 22     & $2 \le N, M \le 1000$ & \\ \hline
  4     & 19     & $2 \le N, M \le 100\,000$ & Enginn vír er þannig að $b < a$. \\ \hline
  5     & 26     & $2 \le N, M \le 100\,000$ & \\ \hline
\end{tabular}

