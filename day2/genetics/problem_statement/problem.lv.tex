\ifx\boi\undefined\ifx\problemname\undefined
\providecommand\sampleinputname{}
\providecommand\sampleoutputname{}
\documentclass[nil]{templates/boi}
\ifdefined\babelprovide
  \babelprovide[import=lv,main]{latvian}
\fi
\problemlanguage{.lv}
\fi
\newcommand{\boi}{Baltijas informātikas olimpiāde}
\newcommand{\practicesession}{Izmēģinājuma kārta}
\newcommand{\contestdates}{27.~aprīlis~-- 1.~maijs, 2018}
\newcommand{\dayone}{1.~diena}
\newcommand{\daytwo}{2.~diena}
\newcommand{\licensingtext}{Šis uzdevums ir licencēts zem CC BY-SA~4.0.}
\newcommand{\problem}{Uzdevums}
\newcommand{\inputsection}{Ievaddati}
\newcommand{\outputsection}{Izvaddati}
\newcommand{\interactivity}{Komunikācija}
\newcommand{\grading}{Testēšana}
\newcommand{\scoring}{Vērtēšana}
\newcommand{\constraints}{Ierobežojumi}
\renewcommand{\sampleinputname}{Ievaddatu paraugs}
\renewcommand{\sampleoutputname}{Izvaddatu paraugs}
\newcommand{\sampleexplanation}[1]{#1.~parauga paskaidrojums}
\newcommand{\sampleexplanations}{Paraugu paskaidrojumi}
\newcommand{\timelimit}{Laika ierobežojums}
\newcommand{\memorylimit}{Atmiņas ierobežojums}
\newcommand{\seconds}{s}
\newcommand{\megabytes}{MB}
\newcommand{\group}{Grupa}
\newcommand{\points}{Punkti}
\newcommand{\limitsname}{Ierobežojumi}
\newcommand{\additionalconstraints}{Papildu ierobežojumi}
\newcommand{\testgroups}{%
Jūsu risinājums tiks testēts uz vairākām testu grupām, par katru no tām var iegūt punktus.
Katra testu grupa satur vienu vai vairākus testus.
Lai iegūtu punktus par testu grupu, jums ir pareizi jāatrisina visi testi šajā grupā.%
}
\fi
\def\version{jury-1}
\problemname{Ģenētika}
Priekš ļaundarēm, kas vēlas pārņemt pasauli, ierasts veids izvairīties no notveršanas
ir sevis klonēšana. Jums izdevās noķert kādu ļaundari un viņas~$N-1$ klonus, un tagad
jūs mēģinat saprast, kura no tām ir īstā ļaundare.

Jūsu rīcībā ir katras personas DNS virkne, kas sastāv no $M$~simboliem, katrs no kuriem ir
\texttt{A}, \texttt{C}, \texttt{G} vai~\texttt{T}.
Jūs arī zinat, ka kloni nav ideāli veidoti; precīzāk, viņu DNS virknes atšķiras tieši $K$~vietās,
salīdzinot ar īstās ļaundares virkni.

Vai jūs varat atpazīt īsto ļaundari?

\section*{\inputsection}
Pirmā rinda satur trīs veselu skaitļus~$N$, $M$ un~$K$, kur $1 \le K \le M$.
Sekojošas $N$~rindas apraksta DNS virknes.
Katra no šīm rindām satur $M$~simbolus; katrs no tiem ir \texttt{A}, \texttt{C}, \texttt{G} vai~\texttt{T}.

Starp ievada virknēm pastāv tieši viena, kas atšķiras no visām pārējām precīzi $K$~vietās.

\section*{\outputsection}
Izvadiet veselu skaitli~--- īstās ļaundares DNS virknes numuru.
Virkņu numerācija sākas ar~$1$.

\section*{\constraints}
\testgroups

\noindent
\begin{tabular}{| l | l | l | l |}
\hline
  \group & \points & \limitsname & \additionalconstraints \\ \hline
  1      & 27      & $2 \le N, M \le 100$ & \\ \hline
  2      & 19      & $2 \le N, M \le 1600$ & Visi simboli ir \texttt{A} vai \texttt{C}. \\ \hline
  3      & 28      & $2 \le N, M \le 4100$ & Visi simboli ir \texttt{A} vai \texttt{C}. \\ \hline
  4      & 26      & $2 \le N, M \le 4100$ & \\ \hline
\end{tabular}
