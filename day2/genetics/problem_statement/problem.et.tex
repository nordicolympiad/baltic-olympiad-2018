\ifx\boi\undefined\documentclass[english]{templates/katt}
\problemlanguage{.en}
\newcommand{\boi}{Baltic Olympiad in Informatics}
\newcommand{\practicesession}{Practice Session}
\newcommand{\contestdates}{April 27 - May 1, 2018}
\newcommand{\dayone}{Day 1}
\newcommand{\daytwo}{Day 2}
\newcommand{\licensingtext}{This problem is licensed under CC BY-SA 4.0.}
\newcommand{\problem}{Problem}
\newcommand{\inputsection}{Input}
\newcommand{\outputsection}{Output}
\newcommand{\interactivity}{Interactivity}
\newcommand{\grading}{Grading}
\newcommand{\scoring}{Scoring}
\newcommand{\constraints}{Constraints}
\newcommand{\sampleinputname}{Sample Input}
\newcommand{\sampleoutputname}{Sample Output}
\newcommand{\sampleexplanation}[1]{Explanation of Sample #1}
\newcommand{\sampleexplanations}{Explanation of Samples}
\newcommand{\timelimit}{Time Limit}
\newcommand{\memorylimit}{Memory Limit}
\newcommand{\seconds}{s}
\newcommand{\megabytes}{MB}
\newcommand{\group}{Group}
\newcommand{\points}{Points}
\newcommand{\limitsname}{Limits}
\newcommand{\additionalconstraints}{Additional Constraints}
\fi
\def\version{jury-1}
\problemname{Geneetika}
Superkurjategijatel on tavaliseks pääsemise viisiks kloonida endast palju koopiaid.
Sul on õnnestunud kätte saada üks selline superkurjategija ja tema $N-1$ klooni,
aga nüüd tuleb sul kindlaks teha, kes neist on originaal.

Et sind selles aidata, on sul iga isiku DNA järjend, mis koosneb $M$ märgist, millest igaüks on kas
\texttt{A}, \texttt{C}, \texttt{G} või \texttt{T}.
On ka teada, et kloonid ei ole perfektsed, vaid nende järjendid erinevad originaalist täpselt 
$K$ märgi osas.

Kas sa suudad leida õige kurjategija?

\section*{\inputsection}
Sisendi esimesel real on kolm täisarvu $N$, $M$ ja $K$, kus $1 \le K \le M$.
Järgmistel $N$ real on DNA järjendid.
Igal real on $M$ märki, mis on kõik kas \texttt{A}, \texttt{C}, \texttt{G} või \texttt{T}.

Täpselt üks sisendis toodud järjenditest on selline, mis erineb kõigist teistest täpselt $K$ märgi osas.

\section*{\outputsection}
Väljastada üks täisarv: esialgsele kurjategijale kuuluva DNA järjendi indeks.
Järjendite nummerdamine algab $1$-st.

\section*{\constraints}
\testgroups

\noindent
\begin{tabular}{| l | l | l | l |}
\hline
  \group & \points & \limitsname & \additionalconstraints \\ \hline
  1      & 27      & $2 \le N, M \le 100$ & \\ \hline
  2      & 19      & $2 \le N, M \le 1600$ & Kõik märgid on kas \texttt{A} või \texttt{C}. \\ \hline
  3      & 28      & $2 \le N, M \le 4100$ & Kõik märgid on kas \texttt{A} või \texttt{C}. \\ \hline
  4      & 26      & $2 \le N, M \le 4100$ & \\ \hline
\end{tabular}
