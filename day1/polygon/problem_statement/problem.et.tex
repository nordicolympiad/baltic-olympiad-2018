\ifx\boi\undefined\documentclass[english]{templates/katt}
\problemlanguage{.en}
\newcommand{\boi}{Baltic Olympiad in Informatics}
\newcommand{\practicesession}{Practice Session}
\newcommand{\contestdates}{April 27 - May 1, 2018}
\newcommand{\dayone}{Day 1}
\newcommand{\daytwo}{Day 2}
\newcommand{\licensingtext}{This problem is licensed under CC BY-SA 4.0.}
\newcommand{\problem}{Problem}
\newcommand{\inputsection}{Input}
\newcommand{\outputsection}{Output}
\newcommand{\interactivity}{Interactivity}
\newcommand{\grading}{Grading}
\newcommand{\scoring}{Scoring}
\newcommand{\constraints}{Constraints}
\newcommand{\sampleinputname}{Sample Input}
\newcommand{\sampleoutputname}{Sample Output}
\newcommand{\sampleexplanation}[1]{Explanation of Sample #1}
\newcommand{\sampleexplanations}{Explanation of Samples}
\newcommand{\timelimit}{Time Limit}
\newcommand{\memorylimit}{Memory Limit}
\newcommand{\seconds}{s}
\newcommand{\megabytes}{MB}
\newcommand{\group}{Group}
\newcommand{\points}{Points}
\newcommand{\limitsname}{Limits}
\newcommand{\additionalconstraints}{Additional Constraints}
\fi
\def\version{jury-1}
\problemname{Armastuse hulknurk}
Nagu me kõik teame, on seebiooperite tegelased omavahel tõsiselt keerulistes suhetes. Igas seebiooperis on $N$ 
tegelast, kellest igaüks armastab täpselt üht isikut. On ka võimalik, et tegelane armastab (alguses) ainult iseennast.

Ütleme, et kaks tegelast on suhtes, kui esimene armastab teist ja teine esimest. Nüüd on meil võimalik lasta kedagi
armunoolega, mille tagajärjel hakkab ta armastama kedagi teist.

Vahel tekib eriti keeruline suhe, mida nimetatakse ``armastuse hulknurgaks''.
Ütleme, et 3 või enam inimest on ``armastuse hulknurgas'', kui esimene armastab teist, teine kolmandat jne,
kuni viimane isik armastab esimest.

Vaatajaküsitlused näitavad, et televaatajad on sellisest draamast väsinud nig eelistaks midagi romantilisemat.
Seega otsustati tulistada osasid tegelasi armunooltega, nii et lõpptulemusena oleksid kõik osalised suhtes.
Mis on minimaalne arv armunooli, mis selleks tuleb kulutada?

\section*{\inputsection}
Esimesel real on täisarv $N$, mis tähistab seebiooperi tegelaste arvu.
Järgmistel $N$ real on igaühel kaks tühikuga eraldatud nime $s$ ja $t$, mis tähendavad, et tegelane $s$ 
armastab alguses tegelast $t$. Nimed on ülimalt $10$ tähe pikkused ja koosnevad ladina tähestiku väiketähtedest.

\section*{\outputsection}
Väljundisse kirjutada üks täisarv -- minimaalne armunoolte arv, mis kindlustaks, et kõik tegelased on suhtes.
Kui see ei ole võimalik, väljastada $-1$.

\section*{\constraints}
\testgroups

\noindent
\begin{tabular}{| l | l | l | l |}
\hline
\group & \points & \limitsname & \additionalconstraints \\ \hline
1     & 21     & $2 \le N \le 20$ & \\ \hline
2     & 25     & $2 \le N \le 100\,000$ & Iga isiku puhul leidub keegi, kes teda armastab. \\ \hline
3     & 29     & $2 \le N \le 100\,000$ & Alguses pole ühtki suhet ega ``armastuse hulknurka''. \\ \hline
4     & 25     & $2 \le N \le 100\,000$ & \\ \hline
\end{tabular}

\section*{\sampleexplanations}

\begin{center}
\includegraphics[width=0.5\textwidth]{polygonfig.pdf}
\end{center}

Esimene näide on toodud joonisel. Ülemine osa illustreerib algset olukorda, kus nool näitab isikult $s$ isikule $t$, 
näidates, et isik $s$ armastab alguses isikut $t$, ning roosa värv tähistab kolme isikut, keda tuleb optimaalse olukorra 
saavutamiseks tulistada. Alumine osa näitab lõppseisu.
%Esimeses näites on ainsaks lahendiks: tulistada
%isikuid \texttt{b}, \texttt{d} ja \texttt{f} armunooltega, mis tähendab, et nad hakkavad armastama vastavalt isikkuid \texttt{h}, \texttt{c} ja \texttt{e}.

Teises näites (mis rahuldab alamülesande 3 tingimusi), on mitu võimalikku lahendust.
Üks võimalus on tulistada iskuid \texttt{a}, \texttt{b} and \texttt{d}, nii et nad hakkavad armastama isikuid \texttt{b}, \texttt{a} ja \texttt{c}.

Kolmandas näites on meil armastuse kolmnurk ja ükskõik, kuidas nooli lasta, jääb keegi alati välja.
