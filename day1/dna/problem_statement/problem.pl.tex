\ifx\boi\undefined\documentclass[english]{templates/katt}
\problemlanguage{.en}
\newcommand{\boi}{Baltic Olympiad in Informatics}
\newcommand{\practicesession}{Practice Session}
\newcommand{\contestdates}{April 27 - May 1, 2018}
\newcommand{\dayone}{Day 1}
\newcommand{\daytwo}{Day 2}
\newcommand{\licensingtext}{This problem is licensed under CC BY-SA 4.0.}
\newcommand{\problem}{Problem}
\newcommand{\inputsection}{Input}
\newcommand{\outputsection}{Output}
\newcommand{\interactivity}{Interactivity}
\newcommand{\grading}{Grading}
\newcommand{\scoring}{Scoring}
\newcommand{\constraints}{Constraints}
\newcommand{\sampleinputname}{Sample Input}
\newcommand{\sampleoutputname}{Sample Output}
\newcommand{\sampleexplanation}[1]{Explanation of Sample #1}
\newcommand{\sampleexplanations}{Explanation of Samples}
\newcommand{\timelimit}{Time Limit}
\newcommand{\memorylimit}{Memory Limit}
\newcommand{\seconds}{s}
\newcommand{\megabytes}{MB}
\newcommand{\group}{Group}
\newcommand{\points}{Points}
\newcommand{\limitsname}{Limits}
\newcommand{\additionalconstraints}{Additional Constraints}
\fi
\def\version{jury-1}
\problemname{Marsjańskie DNA}
Jak pewnie wiesz, ludzkie DNA może być reprezentowane jako długie słowo nad
alfabetem złożonym z czterech liter ({A, C, G, T}), gdzie każdy symbol reprezentuje
różną zasadę azotową nukleotydu (odpowiednio: adeninę, cytozynę, guaninę i tyminę).

W przypadku marsjan sytuacja jest trochę inna; badania prowadzone nad ostatnim
marsjaninem złapanym przez NASA ujawniły, że marsjańskie DNA składa się z oszałamiającej
liczby $K$ różnych zasad nukleotydów! Marsjańskie DNA może zatem być reprezentowane
jako słowo nad alfabetem o rozmiarze $K$.

Teraz pewna grupa badawcza, która jest zainteresowana wykorzystaniem marsjańskiego DNA
w sztucznej inteligencji, zgłosiła zapotrzebowanie na pewien spójny fragment marsjańskiego DNA.
Dla $R$ zasad nukleotydów wymagają oni, aby każda z tych zasad występowała co najmniej pewną liczbę razy w tej próbce.

Chciałbyś znaleźć najkrótsze podsłowo marsjańskiego DNA, które spełnia te wymagania.

\section*{\inputsection}
Pierwszy wiersz zawiera trzy liczby całkowite $N$, $K$ i $R$, oznaczające
odpowiednio długość marsjańskiego DNA, rozmiar alfabetu oraz liczbę zasad nukleotydów,
dla których badacze określili minimalne zapotrzebowanie. Zachodzi $1 \le R \le K \le N$.

Drugi wiersz zawiera opis łańcucha DNA w postaci $N$ liczb całkowitych oddzielonych
pojedynczymi odstępami
Liczba na $i$-tym miejscu, $D_i$, oznacza numer zasady nukleotydu na $i$-tej pozycji w łańcuchu DNA.
Numery zasad są indeksowane od $0$, tj. $0 \leq D_i < K$. Każda z zasad wystąpi w łańcuchu DNA
co najmniej raz.

Każdy z kolejnych $R$ wierszy zawiera po dwie liczby całkowite $B$ i $Q$, oznaczające odpowiednio
numer oraz minimalne zapotrzebowanie na daną zasadę ($0 \le B < K, 1 \le Q \le N$).
Wśród tych $R$ linii żadna zasada nie wystąpi więcej niż raz.

\section*{\outputsection}
Wypisz pojedynczą liczbę całkowitą równą długości najkrótszego spójnego podciągu DNA,
który spełnia warunki naukowców. Jeżeli taki podciąg nie istnieje,
wypisz ,,\texttt{impossible}''.

\section*{\constraints}
\testgroups

\noindent
\begin{tabular}{| l | l | l |}
\hline
\group & \points & \limitsname \\ \hline
1     & 16     & $1 \le N \le 100, R \le 10$ \\ \hline
2     & 24     & $1 \le N \le 4\,000, R \le 10$ \\ \hline
3     & 28     & $1 \le N \le 200\,000, R \le 10$ \\ \hline
4     & 32     & $1 \le N \le 200\,000$ \\ \hline
\end{tabular}

\section*{\sampleexplanations}
W pierwszym przykładzie, mamy trzy spójne podciągi o długości $2$ które zawierają
co najmniej po jednej zasadzie 0 i 1 (tj. ``\texttt{0 1}'', ``\texttt{1 0}'' and ``\texttt{0 1}''),
ale nie ma spójnego podciągu o długości $1$. Najkrótsza długość wynosi zatem $2$.

W drugim przykładzie, optymalny (i jednoznaczny) spójny podciąg to ''{1 3 2 0 1 2 0}''.

W trzecim przykładzie, nie ma wystarczająco dużo zasad typu 0.
