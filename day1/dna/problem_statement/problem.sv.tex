\ifx\boi\undefined\documentclass[swedish]{templates/boi}
\problemlanguage{.sv}
\newcommand{\practicesession}{Övningssession}
\newcommand{\dayone}{Dag 1}
\newcommand{\daytwo}{Dag 2}
\fi
\def\version{jury-1}
\problemname{Marsvarelsers DNA}
Som du antagligen vet kan människans DNA representeras som en lång sträng över ett alfabet med storleken fyra ({A, C, G, T}), där varje bokstav representerar en viss nukleobas (adenin, cytosin, guanin och thymin).

För marsvarelser fungerar det dock lite annorlunda; forskning utförd på den senaste marsvarelsen som NASA fångat avslöjade att marsvarelsers DNA består av inte mindre än $K$ olika nukleobaser! Deras DNA kan alltså representeras som en sträng över ett alfabet med storleken $K$.

Nu har en forskargrupp, som intresserar sig för att utnyttja marsvarelse-DNA för artificiell intelligens, begärt att få en enda sammanhängande delsträng av en marsvarelses DNA-sträng. För $R$ av nukleobaserna har de specificerat 
ett minimalt antal av just den nukleobasen som måste finnas i delsträngen.

Du är intresserad av att hitta den kortaste delsträngen av DNAt som uppfyller deras kriterier.

\section*{\inputsection}
Första raden innehåller tre heltal $N$, $K$ och $R$: den totala längden på marsvarelsens DNA, alfabetets storlek respektive antalet nukleobaser för vilka forskarna har ett minimumkriterium. Talen uppfyller $1 \le R \le K \le N$.

Den andra raden innehåller $N$ blankstegsseparerade heltal, den kompletta DNA-strängen för marsvarelsen. Det $i$-te av dessa heltal, $D_i$, talar om vilken nukleobas som finns på den $i$-te positionen i DNA-strängen. Nukleobaserna är $0$-indexerade, d.v.s.\ de uppfyller $0 \leq D_i < K$. Varje nukleobas förekommer åtminstone en gång.

Var och en av de följande $R$ raderna innehåller två heltal  $B$ and $Q$,  vilka anger en nukleobas och dess minimalt krävda antal.
($0 \le B < K, 1 \le Q \le N$).
Ingen nukleobas kommer att listas mer än en gång.

\section*{\outputsection}
Skriv ut ett enda heltal, längden på den kortaste sammanhängande delsträngen av DNAt som uppfyller forskarnas krav. Om ingen sådan delsträng finns så skriv ut  ``\texttt{impossible}''.

\section*{\constraints}
\testgroups

\noindent
\begin{tabular}{| l | l | l |}
\hline
\group & \points & \limitsname \\ \hline
1     & 16     & $1 \le N \le 100, R \le 10$ \\ \hline
2     & 24     & $1 \le N \le 4\,000, R \le 10$ \\ \hline
3     & 28     & $1 \le N \le 200\,000, R \le 10$ \\ \hline
4     & 32     & $1 \le N \le 200\,000$ \\ \hline
\end{tabular}

\section*{\sampleexplanations}
I det första exemplet finns det tre delsträngar med längden $2$ som innehåller en av varje nukleobas (0 respektive 1), nämligen a ``\texttt{0 1}'', ``\texttt{1 0}'' and ``\texttt{0 1}''),
men inga med längden 1. Alltså är den kortaste längden $2$.

I det andra exemplet är den (unika) optimala delsträngen ``\texttt{1 3 2 0 1 2 0}''.

I det tredje exemplet finns det inte tillräckligt med nukleobaser av typ $0$.

