\ifx\boi\undefined\ifx\problemname\undefined
\providecommand\sampleinputname{}
\providecommand\sampleoutputname{}
\documentclass[icelandic]{templates/boi}
\problemlanguage{.is}
\fi
\newcommand{\boi}{Eystrasaltsólympíuleikarnir í forritnu}
\newcommand{\practicesession}{Æfingakeppni}
\newcommand{\contestdates}{27. apríl - 1. maí, 2018}
\newcommand{\dayone}{Dagur 1}
\newcommand{\daytwo}{Dagur 2}
\newcommand{\licensingtext}{This problem is licensed under CC BY-SA 4.0.}
\newcommand{\problem}{Verkefni}
\newcommand{\inputsection}{Inntak}
\newcommand{\outputsection}{Úttak}
\newcommand{\interactivity}{Gagnvirkni}
\newcommand{\grading}{Flokkun}
\newcommand{\scoring}{Stigagjöf}
\newcommand{\constraints}{Takmarkanir}
\renewcommand{\sampleinputname}{Sýnidæmis inntak}
\renewcommand{\sampleoutputname}{Sýnidæmis úttak}
\newcommand{\sampleexplanation}[1]{Útskýring á sýnidæmi #1}
\newcommand{\sampleexplanations}{Útskýringar á sýnidæmum}
\newcommand{\timelimit}{Tímamörk}
\newcommand{\memorylimit}{Minnismörk}
\newcommand{\seconds}{s}
\newcommand{\megabytes}{MB}
\newcommand{\group}{Hópur}
\newcommand{\points}{Stig}
\newcommand{\limitsname}{Takmarkanir}
\newcommand{\additionalconstraints}{Auka takmarkanir}
\newcommand{\testgroups}{
Lausnin þín verður prófuð á einhvern fjölda prufuhópa, hver hópur gefur einhvern fjölda stiga.
Hver hópur inniheldur einhvern fjölda prufutilvika.
Til að fá stig fyrir hóp þarftu að leysa öll prufutilvik innan hópsins.
}
\fi
\def\version{jury-1}
\problemname{Marsneskt erfðaefni}
Eins og þú líklega veist að þá má tákna mennskt erfðaefni með löngum streng yfir
stafrófið ({A, C, G, T}), þar sem hvert tákn vísar í mismunandi basa 
(hvert um sig í þeirri röð sem um var getið; adenín, sýtósín, gúanín og týmín).

Fyrir marsbúa, hinsvegar, eru hlutirnir aðeins öðruvísi. Rannsóknir sem voru
framkvæmdar á þeim marsbúa sem var nýlegast fangaður af NASA sýndi að marsneskt erfðaefni
samanstendur af heilum $K$ mismunandi bösum! Marsneskt erfðaefni má því tákna með streng
yfir $K$ tákna stafróf.

Nú hefur sérstakur rannsóknarhópur, sem hefur áhuga á að nýta marsneskt erfðaefni í 
gervigreind, beðið um að fá einn samliggjandi bút úr streng af marsnesku erfðaefni.
Fyrir $R$ basanna, hafa þeir skilgreind minnsta magn af þeim basa sem þarf að
koma fyrir í sýninu þeirra.

Þú hefur áhuga á að finna stysta hlutstreng í erfðaefninu sem uppfyllir skilyrðin.

\section*{\inputsection}
Fyrsta línan inniheldur þrjár heiltölur $N$, $K$ og $R$ sem tákna samtals
lengdina á marsneska erfðaefninu, stærðina á stafrófinu og fjölda basa sem
rannsakendurnir skilgreindu skilyrði fyrir. Þær uppfylla $1 \le R \le K \le N$.

Næsta lína inniheldur $N$ heiltölur aðskildar með bili, sem tákna marsneska
erfðaefnis strenginn. Heiltala númer $i$, $D_i$, táknar hvaða basi er í staðsetningu
$i$ í strengnum. Basar eru $0$-vísasettir, þ.e. $0 \leq D_i < K$. Hver einasti 
basi mun koma allavega einu sinni fyrir í erfðaefnis strengnum.

Hver og ein af næstu $R$ línum inniheldur tvær heiltölur $B$ og $Q$ sem tákna basa
og minnsta fjölda sem þarf af þeim basa í sýninu ($0 \le B < K, 1 \leq Q \le N$).
Hver basi mun ekki koma fyrir oftar en einu sinni í þessum $R$ línum.

\section*{\outputsection}
Skrifaðu út eina heiltölu, lengdina á stysta samliggjandi hlutsreng af erfðaefni
sem uppfyllir skilyrði rannsakendanna. Ef ekki er til hlutstrengur sem uppfyllir
skilyrðin þá skaltu skrifa út ``\texttt{impossible}''.

\section*{\constraints}
\testgroups

\noindent
\begin{tabular}{| l | l | l |}
\hline
\group & \points & \limitsname \\ \hline
1     & 16     & $1 \le N \le 100, R \le 10$ \\ \hline
2     & 24     & $1 \le N \le 4\,000, R \le 10$ \\ \hline
3     & 28     & $1 \le N \le 200\,000, R \le 10$ \\ \hline
4     & 32     & $1 \le N \le 200\,000$ \\ \hline
\end{tabular}

\section*{\sampleexplanations}
Í fyrsta sýnidæminu eru þrír hlutstrengir af lengd $2$ sem innihalda einn af hvorum basa en
engir hlutstrengir af lengd $1$. Því er stysta lengdin $2$.

Í öðru sýnidæminu er aðeins einn besti hlutstrengur, ``\texttt{1 3 2 0 1 2 0}''.

Í þriðja sýnidæminu eru ekki nógu margir basar af týpu 0.
