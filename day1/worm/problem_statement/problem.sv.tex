\ifx\boi\undefined\documentclass[swedish]{templates/boi}
\problemlanguage{.sv}
\newcommand{\practicesession}{Övningssession}
\newcommand{\dayone}{Dag 1}
\newcommand{\daytwo}{Dag 2}
\fi
\def\version{jury-1}
\problemname{Mask-omsorg}

Du letar efter en plats i jorden att lägga din tama mask, Maximus. Du begränsar ditt letande till ett rätblock med dimensionerna $N \times M \times K$ centimeter, vilket du har delat in i ett tredimensionellt rutnät av kubikcentimeter-celler, benämnda $(x,y,z)$ enligt deras position i rutnätet ($1 \le x \le N$, $1 \le y \le M$, $1 \le z \le K$). Varje cell har en viss fuktighet $H[x,y,z]$ som är ett heltal mellan $1$ och $10^9$. Du kan mäta fuktigheten av valfri cell med en speciell sensor.

Maximus älskar fuktiga platser, så du måste placera honom i en cell som är minst lika fuktig som dess angränsande celler, annars kravlar han iväg och du får problem att hitta honom. Med andra ord ska du placera Maximus i ett lokalt maximum.
Mer preciserat: du ska finna en cell $(x,y,z)$, sådan att
$$
H[x,y,z] \ge \max(H[x+1,y,z], H[x-1,y,z], H[x,y+1,z], H[x,y-1,z], H[x,y,z+1], H[x,y,z-1]),
$$
där ett värde anses som $0$ om det är utanför rätblocket (eftersom Maximum absolut vill stanna inom detta).

Antalet celler kan dock vara väldigt stort, så du vill inte mäta fuktigheten på alla celler. Därför kan du i den här uppgiften interagera med Kattis och fråga efter fuktigheten i givna punkter. När du har hittat en lämplig placering åt Maximus, ange denna plats till Kattis.

\section*{\interactivity}
The first line of the input contains four positive integers: $N$, $M$, $K$ and $Q$, where $N$, $M$ and $K$ are the box dimensions and $Q$ is the maximum number of measurements you may perform.

After that, you can write at most $Q$ lines of the form \texttt{?\ x y z} to standard output.
This asks for the value of the humidity at point $(x, y, z)$.
For each such line, the grader will in response write a single line with the integer $H[x,y,z]$, which can be read from standard input by your program.

After all these lines, your program must write out exactly one line of the form \texttt{!\ x y z}.
This claims that the point $(x, y, z)$ is a suitable location for Maximus according to the criterion above.
The grader will provide no response to this output.

All values of $x, y, z$ must obey $1 \le x \le N$, $1 \le y \le M$, $1 \le z \le K$.
If they do not, or some line has an invalid format, or you ask for more than $Q$ values,
the grader will respond with \texttt{-1} and exit.
If this happens your program should also exit. If it continues, it may incorrectly get
a verdict of Runtime Error or Time Limit Exceeded.

You \emph{must} make sure to flush standard output before reading the grader's response, or else your program
will get judged as Time Limit Exceeded. This works as follows in the supported languages:
\begin{itemize}
  \item Java: \texttt{System.out.println()} flushes automatically.
  \item Python: \texttt{print()} flushes automatically.
  \item C++: \texttt{cout << endl;} flushes, in addition to writing a newline. If using printf, \texttt{fflush(stdout)}.
  \item Pascal: \texttt{Flush(Output)}.
\end{itemize}

To help deal with this interaction, we provide optional helper code which you may copy into your program.
A link to this code for all supported languages (C++, Pascal, Java, Python) can
be found in the sidebar of the kattis problem page.
We especially recommend using this code if using Java or Python, whose default
input/output (IO) routines \emph{may not be fast enough for the last two test groups}.
The helper code uses optimized IO routines which are sufficiently fast.

The grader will be \emph{non-adaptive}; that is, each test case will have a fixed set of humidity values
that do not depend on what measurements are performed by the program.

\section*{\constraints}
\testgroups

\noindent
\begin{tabular}{| l | l | l |}
\hline
\group & \points & \limitsname \\ \hline
1      & 10     & $M = K = 1$, $N = 1\,000\,000$, $Q = 10\,000$  \\ \hline
2      & 22     & $M = K = 1$, $N = 1\,000\,000$, $Q = 35$       \\ \hline
3      & 12     & $K = 1$, $N = M = 200$,         $Q = 4\,000$   \\ \hline
4      & 19     & $K = 1$, $N = M = 1\,000$,      $Q = 3\,500$   \\ \hline
5      & 14     & $N = M = K = 100$,              $Q = 100\,000$ \\ \hline
6      & 23     & $N = M = K = 500$,              $Q = 150\,000$ \\ \hline
\end{tabular}
