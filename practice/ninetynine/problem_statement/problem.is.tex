\ifx\boi\undefined\ifx\problemname\undefined
\providecommand\sampleinputname{}
\providecommand\sampleoutputname{}
\documentclass[icelandic]{templates/boi}
\problemlanguage{.is}
\fi
\newcommand{\boi}{Eystrasaltsólympíuleikarnir í forritnu}
\newcommand{\practicesession}{Æfingakeppni}
\newcommand{\contestdates}{27. apríl - 1. maí, 2018}
\newcommand{\dayone}{Dagur 1}
\newcommand{\daytwo}{Dagur 2}
\newcommand{\licensingtext}{This problem is licensed under CC BY-SA 4.0.}
\newcommand{\problem}{Verkefni}
\newcommand{\inputsection}{Inntak}
\newcommand{\outputsection}{Úttak}
\newcommand{\interactivity}{Gagnvirkni}
\newcommand{\grading}{Flokkun}
\newcommand{\scoring}{Stigagjöf}
\newcommand{\constraints}{Takmarkanir}
\renewcommand{\sampleinputname}{Sýnidæmis inntak}
\renewcommand{\sampleoutputname}{Sýnidæmis úttak}
\newcommand{\sampleexplanation}[1]{Útskýring á sýnidæmi #1}
\newcommand{\sampleexplanations}{Útskýringar á sýnidæmum}
\newcommand{\timelimit}{Tímamörk}
\newcommand{\memorylimit}{Minnismörk}
\newcommand{\seconds}{s}
\newcommand{\megabytes}{MB}
\newcommand{\group}{Hópur}
\newcommand{\points}{Stig}
\newcommand{\limitsname}{Takmarkanir}
\newcommand{\additionalconstraints}{Auka takmarkanir}
\newcommand{\testgroups}{
Lausnin þín verður prófuð á einhvern fjölda prufuhópa, hver hópur gefur einhvern fjölda stiga.
Hver hópur inniheldur einhvern fjölda prufutilvika.
Til að fá stig fyrir hóp þarftu að leysa öll prufutilvik innan hópsins.
}
\fi
\def\version{jury-1}
\problemname{Níutíu og níu}

Þú ert að spila leik með vini þínum sem kallast \emph{Níutíu og níu}.
Þú byrjar með því að segja annaðhvort töluna $1$ eða töluna $2$.
Þið skiptist svo á að gera í umferðum og hækkið töluna um annaðhvort $1$ eða $2$ í hverri umferð.
Fyrsti leikmaður til að segja töluna $99$ vinnur.

Skrifaðu forrit sem spilar leikinn fyrir þig og vinnur.

\section*{\interactivity}
Þetta verkefni er gagnvirkt.

Forritið þitt skal byrja á að skrifa út annaðhvort $1$ eða $2$ í einni línu.
Yfirferðarforritið les síðan þessa tölu (köllum hana $x$), og skrifar á móti eina línu með annaðhvort $x+1$ eða $x+2$, sem forritið þitt getur þá lesið.
Forritið þitt skal síðan skrifa út tölu sem er $1$ eða $2$ hærri, og svo framvegis.

Ef þér tekst að sigra og skrifa út $99$, skal forritið þitt hætta keyrslu eðlilega (skila út 0).
Hins vegar ef forritið þitt les $99$ skal það einnig hætta keyrslu eðlilega, en þá fær það niðurstöðuna Wrong Answer.
Að skrifa út ógild gildi (þar með taldnar tölur stærri en $99$) mun einnig valda niðurstöðunni Wrong Answer, gefið að forritið þitt hætti keyrslu að lokum.
Ef forritið þitt hættir keyrslu óeðlilega, mun það valda Wrong Answer, Runtime Error eða Time Limit Exceeded eftir því hvað á við.

Þú \emph{þarft} að passa að sturta úttakinu út áður en þú lest svarið frá yfirferðarforritinu,
annars mun forritið þitt fá niðurstöðuna Time Limit Exceeded. Þetta virkar í studdum málum á eftirfarandi hátt:
\begin{itemize}
  \item Java: \texttt{System.out.println()} sturtar sjálfkrafa.
  \item Python: \texttt{print()} sturtar sjálfkrafa.
  \item C++: \texttt{cout << endl;} sturtar, og fer einnig í næstu línu. Ef printf er notað, \texttt{fflush(stdout)}.
  \item Pascal: \texttt{Flush(Output)}.
\end{itemize}

\section*{\constraints}
\testgroups

\noindent
\begin{tabular}{| l | l | l |}
\hline
\group & \points & \constraints \\ \hline
  1      & 30     & Vinur þinn hækkar töluna alltaf um $1$. \\ \hline
  2      & 30     & Vinur þinn hækkar töluna alltaf um $2$ (nema þegar talan er 98). \\ \hline
  3      & 40     & Vinur þinn spilar handahófskennt, þar sem hvor möguleiki er spilaður með 50\% líkum (nema talan sé 98). \\ \hline
\end{tabular}
