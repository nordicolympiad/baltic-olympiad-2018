\ifx\boi\undefined\documentclass[english]{templates/katt}
\problemlanguage{.en}
\newcommand{\boi}{Baltic Olympiad in Informatics}
\newcommand{\practicesession}{Practice Session}
\newcommand{\contestdates}{April 27 - May 1, 2018}
\newcommand{\dayone}{Day 1}
\newcommand{\daytwo}{Day 2}
\newcommand{\licensingtext}{This problem is licensed under CC BY-SA 4.0.}
\newcommand{\problem}{Problem}
\newcommand{\inputsection}{Input}
\newcommand{\outputsection}{Output}
\newcommand{\interactivity}{Interactivity}
\newcommand{\grading}{Grading}
\newcommand{\scoring}{Scoring}
\newcommand{\constraints}{Constraints}
\newcommand{\sampleinputname}{Sample Input}
\newcommand{\sampleoutputname}{Sample Output}
\newcommand{\sampleexplanation}[1]{Explanation of Sample #1}
\newcommand{\sampleexplanations}{Explanation of Samples}
\newcommand{\timelimit}{Time Limit}
\newcommand{\memorylimit}{Memory Limit}
\newcommand{\seconds}{s}
\newcommand{\megabytes}{MB}
\newcommand{\group}{Group}
\newcommand{\points}{Points}
\newcommand{\limitsname}{Limits}
\newcommand{\additionalconstraints}{Additional Constraints}
\fi
\def\version{jury-1}
\problemname{Viited}
\illustration{.40}{oldbooks}{\href{https://pixabay.com/en/books-old-books-antiquariat-read-1215672/}{CC0, congerdesign via Pixabay.}}

Grace tahab lugeda üht teadusraamatut.
Tal on harjumus alati hoolikalt lugeda ka kõiki allikaid, millele tema loetavad raamatud viitavad ja nende allikaid j.n.e.
Tavaliselt on tulemuseks, et ta loeb märksa rohkem raamatuid kui see üks, mida ta algselt lugeda plaanis.
Raamatukoguhoidjad juba pahandavad tema tohutute laenutusmahtude pärast.
Selle vähendamiseks tahab ta leida, mis järjekorras ta peaks raamatuid lugema, et summaarne laenutusaeg oleks vähim võimalik.

On teada, et lõpuks loeb ta kokku $N$ raamatut.
Raamatud on nummerdatud $1 \dots N$ ja raamatu $i$ lugemiseks kulub $K_i$ minutit.
Raamatus $i$ on viited $F_i$ teisele raamatule.
Raamat, mida Grace algselt lugeda plaanis, on number $1$.
Enne lugemise algust on Grace juba laenutanud kõik $N$ raamatut.

Iga raamatu lugemisel käitub Grace järgmiselt:
\begin{itemize}
\item Kõigepealt avab ta raamatu ja loeb läbi selle viidete nimekirja, milleks kulub üks minut.
\item Siis loeb ta mingis tema valitud järjekorras läbi kõik viidatud raamatud.
\item Lõpuks loeb ta läbi raamatu enda ja tagastab selle raamatukokku, milleks kulub $K_i$ minutit.
\end{itemize}

Leida minimaalne laenutuste aegade summa, kui Grace loeb raamatuid optimaalses järjekorras.
Mõne raamatu viidete nimekiri võib olla tühi ja on teada, et iga raamat (peale raamatu $1$) on täpselt ühe teise raamatu viidete nimekiras.
Lisaks on teada, et viidete hulgas ei ole tsükleid.

\section*{\inputsection}
Sisendi esimesel real on raamatute arv $N$ ($1 \le N \le 100\,000$).
Järgmised $N$ rida kirjeldavad raamatuid nende numbrite järjekorras.
Igal real on kõigepealt raamatu lugemiseks kuluvate minutite arv $K_i$ ($1 \le K_i \le 1\,000$),
selle järel raamatus olevate viidete arv $F_i$ ($0 \le F_i < N$),
ja selle järel $F_i$ viidatavate raamatute numbrit.

\section*{\outputsection}
Väljastada täpselt üks täisarv, vähim võimalik kõigi raamatute laenutusaegade summa.

\section*{\constraints}
\testgroups

\noindent
\begin{tabular}{| l | l | l |}
\hline
\group & \points & \limitsname \\ \hline
1     & 20     & $1 \le N \le 10, 1 \le K_i \le 1000$ \\ \hline
2     & 30     & $1 \le N \le 50, 1 \le K_i \le 10$, $F_i \le 5$ \\ \hline
3     & 20     & $1 \le N \le 100\,000, 1 \le K_i \le 1000, F_i \le 5$ \\ \hline
4     & 30     & $1 \le N \le 100\,000, 1 \le K_i \le 1000$ \\ \hline
\end{tabular}

\section*{\sampleexplanation{1}}
1. minut: ava raamat 1 \\
2. minut: ava raamat 2 \\
3. minut: ava raamat 4 \\
4. minut: tagasta raamat 4 (laenutuse aeg 4 minutit) \\
14. minut: tagasta raamat 2 (laenutuse aeg 14 minutit) \\
15. minut: ava raamat 3 \\
16. minut: ava raamat 5 \\
17. minut: tagasta raamat 5 (laenutuse aeg 17 minutit) \\
37. minut: tagasta raamat 3 (laenutuse aeg 37 minutit) \\
38. minut: tagasta raamat 1 (laenutuse aeg 38 minutit) \\
